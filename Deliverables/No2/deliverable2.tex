\documentclass[a4paper,11pt]{article}

% Use utf-8 encoding for foreign characters
\usepackage[utf8]{inputenc}
\usepackage[british,english]{babel}
\usepackage[T1]{fontenc}

% Setup for fullpage use
\usepackage{fullpage}

% Multipart figures
\usepackage{subfigure}

% More symbols
\usepackage{amsmath}
\usepackage{amssymb}
\usepackage{latexsym}

% For pretty URLs, see: http://en.wikibooks.org/wiki/LaTeX/Hyperlinks
\usepackage{hyperref}

% Surround parts of graphics with box
\usepackage{boxedminipage}

% Package for including code in the document
\usepackage{listings}

% If you want to generate a toc for each chapter (use with book)
\usepackage{minitoc}

% Uncomment if you want to use Palatino as font
\usepackage[sc]{mathpazo}
\linespread{1.05}         % Palatino needs more leading (space between lines)

% This is now the recommended way for checking for PDFLaTeX:
\usepackage{ifpdf}

%\newif\ifpdf
%\ifx\pdfoutput\undefined
%\pdffalse % we are not running PDFLaTeX
%\else
%\pdfoutput=1 % we are running PDFLaTeX
%\pdftrue
%\fi

\ifpdf
\usepackage[pdftex]{graphicx}
\else
\usepackage{graphicx}
\fi
\title{Product and Process Brief\\\small{for}\\\small{Danske Bank: Peer-to-peer}}
\author{ Jesper Borgstrup, Thomas Kjeldsen and Mads Ohm Larsen }

\date{Updated: March 1st}

\begin{document}

\ifpdf
\DeclareGraphicsExtensions{.pdf, .jpg, .tif}
\else
\DeclareGraphicsExtensions{.eps, .jpg}
\fi

\maketitle

\tableofcontents
\vspace{2cm}

% \section{temporary notes / todos}
%læst op på bluetooth API
%undersøg nok med bluetooth encryption?
%bump overførsel af ???: secretkey + bluetooth UUID

%Refs:
% > http://developer.android.com/reference/javax/crypto/interfaces/package-summary.html
% Diffie-Helman cryptography

%> http://developer.android.com/sdk/android-2.3.3.html
%Bluetooth: Android 2.3.3 adds platform and API support for Bluetooth nonsecure socket connections. This lets applications communicate with simple devices that may not offer a UI for authentication. See createInsecureRfcommSocketToServiceRecord(java.util.UUID) and listenUsingInsecureRfcommWithServiceRecord(java.lang.String, java.util.UUID) for more information.

%> http://mobisocial.stanford.edu/news/2011/02/nfc-and-bluetooth-brought-together-with-android-2-3-3/
%To use NFC to kick off a Bluetooth session, first start a running socket server on one device, using a randomly generated service UUID. Then, create an NDEF message encoding the MAC address of the phone’s Bluetooth device and the UUID of the listening service. When this message is received by the joining phone, it will have enough information to connect to the Bluetooth server socket

% end temporary notes

\pagebreak 

\section{Purpose/Vision} % (fold)
\label{sec:purpose_vision}

The aim of this project is to provide a library, which Android App developers can utilize to facilitate easy and secure communication between two phones over Bluetooth.

Providing secure communication can be reduced to two discting problems: 1) securing communication against eavesdropping and 2) validating the identity of the other party to protect against man-in-the-middle style attacks.

We will use Bluetooth as the transport medium over which the transfers take place, which in addition to cryptographic measures adds additional resilience by forcing a potential attacker to be in close proximity to the user.
Bluetooth offers advantages compared to transmitting data over the mobile network, since it is is quite efficient in terms of both power drain and transfer speed and it is not subject to usage fees. 

The library will be able to support multiple identity providers, which provide identification of the client and transport the neccessary information (out of band) for setting up the secure bluetooth transport. Support for using Bump as an identity provider will be included as a part of the project, but the library could be extended with further identity providers such as NFC (Near Field Communication) or 3rd party solutions e.g. a custom authentification server set up by a Bank, by NemID etc.

One of the disadvantages of bluetooth is that it is not that convenient to connect two phones: both phones have to enable the bluetooth radio, one has to be in discoverable mode (at least temporarily) and the users need to scan and pick the right device to transmit to. Easy setup of a bluetooth connection between two phones is one of the things NFC is poised to facilitate
\footnote{
\url{http://www.eetimes.com/design/communications-design/4012606/How-NFC-can-to-speed-Bluetooth-transactions-151-today}
}, but with support for Bump as an identity provider this library will facilitate easy connection setup for the large number of phones that do not have a built-in NFC chip.\\

%value proposition:
%powerconsumption per kb?
%* bluetooth
%* gprs/3g

\noindent
Our primary goal with this project is the library described above and the capabilities it provides, but to showcase the use of the library another part of the project will be to develop a sample application. Our sample application will focus on facilitating secure file transfers.

\subsection{Roadmap} % (fold)
\label{subsec:roadmap}
Our roadmap is divided into our five sprints. These five sprints, and their goals, are:
\begin{itemize}
	\item Sprint \#1 (March 11$^{\text{th}}$, 2011)
	\begin{itemize}
		\item Read up on relevant documentation, APIs and product and process brief refined. Implementation of a fully working prototype that allows two Android smartphones to be bumped together and transfer information (specifically bluetooth MAC + UUID, neccesary for setting up bluetooth connection in next sprint) between them.
	\end{itemize}
\pagebreak
	\item Sprint \#2 (April 1$^{\text{st}}$, 2011)
	\begin{itemize}
		\item Library can perform secure bluetooth connection setup. Sample App can perform secure transfers between two phones and supports easy initiation directly from a file's context menu. 
	\end{itemize}
	
	\item Sprint \#3 (April 29$^{\text{th}}$, 2011)
	\begin{itemize}
		\item API finalised. Code is robust, well tested and documented, consistent and implements alle current Android Best Practices. Library completed, packaged and published on Android Market.
	\end{itemize}
	
	\item Sprint \#4 (May 20$^{\text{th}}$, 2011)
	\begin{itemize}
		\item Additional identity provider included in library: NFC. Library targets and supports multiple Android API levels. Sample App supports sharing using NFC.
	\end{itemize}
	
	\item Sprint \#5 (June 10$^{\text{th}}$, 2011)
	\begin{itemize}
		\item Wrap up.
		% Time allowing perhaps an additional extension:
		% * safe chat
		% * 3rd party identity provider example (bank, PKI infrastructure w/ client certificates, ...)
		% * Investigate Root User possibilities:
		% * * even less confirmation dialogs etc
		% * * join ad-hoc wireless or transfer using
	\end{itemize}
\end{itemize}
% subsection roadmap (end)

% section purpose_vision (end)

\section{User/Customer} % (fold)
\label{sec:user_customer}

Our customers are Android app developers, who are interested in transferring information between two phones in a both easy and secure way. Especially apps where the need for confidentiality is so high that US based companies such as Bump and their privacy policy are not enough (e.g. financial institutions), or where the amount of information is so large that the cellular network is not the best option.

Sune Lomholt has agreed to act as a representative of our product owner(s).

% section user_customer (end)

\section{Team resources, roles and obligations} % (fold)
\label{sec:team_resources_roles_and_obligations}
The team consists of three Master's students at the Department of Computer Science of the University of Copenhagen (DIKU): \\

\begin{tabular}{|p{4.5cm}|p{5cm}|p{3.5cm}|}
\hline
\textbf{Name}    & E-mail				          &	Telephone         \\\hline
Jesper Borgstrup & \href{mailto:jesper@borgstrup.dk}{jesper@borgstrup.dk} 	  & (+45) 61 30 30 81 \\\hline
Thomas Kjeldsen  & \href{mailto:thomas@thomaskjeldsen.dk}{thomas@thomaskjeldsen.dk} & (+45) 61 30 80 01 \\\hline
Mads Ohm Larsen  & \href{mailto:mads.ohm@gmail.com}{mads.ohm@gmail.com} & (+45) 60 16 39 53 \\\hline
\end{tabular}

\subsection{Skills} % (fold)
\label{subsec:skills}
In our team we will need some different skills.
This is the distribution on how we find ourselves fitted for these skills: \\

\begin{tabular}{|p{4.5cm}|p{3cm}|p{3cm}|p{3cm}|}
\hline
\textbf{Skills}          & \textbf{High}   & \textbf{Medium} & \textbf{Low} 					 \\\hline
Android development      & Jesper & Thomas & Mads 					 \\\hline
Configuration Management & Mads & Jesper and Thomas          &      					 \\\hline
Scrum Master             &        &        & Jesper, Thomas and Mads \\\hline 
User involvement         &        & Mads and Jesper & Thomas \\\hline
Test                     & Mads and Thomas   & Jesper & \\\hline
\end{tabular}
% subsection skills (end)

\subsection{Resources} % (fold)
\label{subsec:resources}
Correspondence between team members and roles:\\

\begin{tabular}{|p{3cm}|p{5cm}|p{5cm}|}
\hline
\textbf{Team Member} & \textbf{Primary roles and responsibilities}   & \textbf{Strengths and focus area} \\\hline
Jesper               & Coder                   & Development \\\hline
Thomas               & Scrum Master, coder, tester    & Test                 \\\hline
Mads                 & SCM maintainer, coder, tester  & Test, user involvement                         \\\hline 
\end{tabular}
\\
% TODO: Overvej ovenstående
\\
These roles are very loose, and will be subject to change throughout the course. E.g. we will all be Scrum Master at some point during this course.
% subsection resources (end)

% section team_resources_roles_and_obligations (end)

\section{Team empowerment} % (fold)
\label{sec:team_empowerment}
\begin{itemize}
	\item The team breaks down and estimates prioritised work items in the sprint backlog

	\item The team jointly determines how to perform the work, including possible ad hoc planning meetings

	\item Each team member plans their own daily work in respect for planned activities/meeting
\end{itemize}

% section team_empowerment (end)

\section{Team values} % (fold)
\label{sec:team_values}
The following are agreements made between the project members

\begin{itemize}
	\item We're respectfully nice to each other.

  	\item If someone's feeling overwhelmed they're entitled to a hug (subject to a daily maximum of 3 and a weekly maximum of 5)

  	\item We show up on time, and if we're prevented or running late we let each other know.

  	\item We speak openly about problems

  	\item We share information, be that information from or to product owner, or between us, via e-mail, the configuration management system or other ways

  	\item We participate equally, meaning that everybody have an equal work load
\end{itemize}

% section team_values (end)

\section{Team processes} % (fold)
\label{sec:team_processes}

\begin{itemize}
	\item Our code will be open for everybody to see
	
	\item We have agreed on using git, and our code will be stationed at GitHub
	
	\item We have agreed on working together in person, but also being available on both a messenger platform and on Skype - or eMeeting (a platform suggested by Sune Lomholt)
\end{itemize}

\subsection{Configuration management} % (fold)
\label{subsec:configuration_management}

We use Git as our SCM system. 
Our repository is hosted publicly on GitHub.

We're using a Test Driven Development approach, which imposes the following structure on the \texttt{master} branch:
\begin{itemize}
\item Code committed to the central repository must be thoroughly covered by JUnit tests.
\item Tests must not fail when comitted, even if this means methods are stubbed.
\item Code must compile.
\end{itemize}

% subsection configuration_management (end)

\subsection{Office rules} % (fold)
\label{subsec:office_rules}
When working together, e.g. sitting together and working on stories, the following should be respected:

\begin{itemize}
	\item No loud music or noises

	\item Procrastination should not affect others
\end{itemize}
% subsection office_rules (end)

\subsection{Calendar planning} % (fold)
\label{subsec:calendar_planning}

We have agreed to regularly do a virtual scrum logging, where each participant writes an entry firstly about what he did today, what he will do tomorrow, and if any problems have occurred.

Sprint demos will occur at, or around, the ending of a sprint, depending on when the product owner have got time.
At the time of sprint demo, we will properly want to do a sprint planning, for the following sprint, together with the product owner.

At the end of a sprint, we will have a so-called sprint retrospective, discussing what went well and what could be approved for next sprint.

We use a shared Google calendar for recurring meetings and other scheduled events.

% subsection calendar_planning (end)

% section team_processes (end)

\section{Team performance \& progress monitoring} % (fold)
\label{sec:team_performance_progress_monitoring}
To support our process we are interested in a scrum tool that, especially because we have no offices or static whiteboards, facilitates a virtual collaboration using the scrum model.

We are using Acunote as our scrum tool and have set up a logbook for our (virtual) scrum meetings, a burndown chart, sprint- and product backlogs.

Throughout Sprint 1 we will will assess how the tools are working to ensure that they are being used to their maximum potential and they are providing the neccesary product support.

% TODO: Der skal skrives at vi bruger Acunote?

% section team_performance_progress_monitoring (end)


\end{document}
