\documentclass[a4paper,11pt]{article}

% Use utf-8 encoding for foreign characters
\usepackage[utf8]{inputenc}
\usepackage[british,english]{babel}
\usepackage[T1]{fontenc}

% Setup for fullpage use
\usepackage{fullpage}

% Multipart figures
\usepackage{subfigure}

% More symbols
\usepackage{amsmath}
\usepackage{amssymb}
\usepackage{latexsym}

% For pretty URLs, see: http://en.wikibooks.org/wiki/LaTeX/Hyperlinks
\usepackage{hyperref}

% Surround parts of graphics with box
\usepackage{boxedminipage}

% Package for including code in the document
\usepackage{listings}

% If you want to generate a toc for each chapter (use with book)
\usepackage{minitoc}

% Uncomment if you want to use Palatino as font
\usepackage[sc]{mathpazo}
\linespread{1.05}         % Palatino needs more leading (space between lines)

% This is now the recommended way for checking for PDFLaTeX:
\usepackage{ifpdf}

%\newif\ifpdf
%\ifx\pdfoutput\undefined
%\pdffalse % we are not running PDFLaTeX
%\else
%\pdfoutput=1 % we are running PDFLaTeX
%\pdftrue
%\fi

\ifpdf
\usepackage[pdftex]{graphicx}
\else
\usepackage{graphicx}
\fi
\title{Deliverable 3: Sprint \#1\\\small{for}\\\small{Danske Bank: Peer-to-peer}}
\author{ Group Delta:\\Jesper Borgstrup, Thomas Kjeldsen and Mads Ohm Larsen }

\date{March 11, 2011}

\begin{document}

\ifpdf
\DeclareGraphicsExtensions{.pdf, .jpg, .tif}
\else
\DeclareGraphicsExtensions{.eps, .jpg}
\fi

\maketitle

%\tableofcontents
%\vspace{2cm}

\section{Requirements for this deliverable}
\begin{enumerate}
\item Doing a demo in class (on 2011-03-09)
\item Giving us access to your source code
\item Handing in a collection of your sprint material
\item Describing a sprint retrospective (e.g., as a set of bullets outlining what
when well, what went wrong, and how you will improve for the next sprint)
\end{enumerate}

The Sprint Demo was given on March 9th. This document describes requirements 2-4.

\section{Source code access}
Our source code is publicly available on Github from \url{https://github.com/omegahm/DBP2P}.

Please note that we are working on multiple branches (use the button switch branch to view another branch).

The master branch currently holds only documentation and deliverables, while we have a dedicated development branch for sprint \#1 named ``Dustytuba''.

If you wish to checkout our code (read-only) using Git, then use git clone with this URL:
\url{git://github.com/omegahm/DBP2P.git}


\section{Sprint material}
%TODO
Sprint Material needed to assess our progress include the following:
\begin{itemize}
\item source code (version number and access method is sufficient)
\item product backlog (before and after the sprint)
\item sprint backlog
\item any other material (e.g., burndown chart) that illustrates your progress
\end{itemize}

% TODO: Describe Acunote in general and that access have been granted.

\subsection{Source Code}
%TODO: Merge final product, or rewrite
%TODO: Update commit ID + URL
The final product of sprint \#1 has been merged into the master branch as per commit-id  \href{https://github.com/omegahm/DBP2P/commit/0a8382e920d5129e4e57a7e54018b8769dcb3273}{0a8382e920d5129e4e57}.

The code and tests are located in the folders ``DustyTuba'' and ``DustyTuba''.

\subsection{Product Backlog (before sprint)}
%TODO: Nothing

\subsection{Product Backlog (after sprint)}
%TODO: Insert

\subsection{Sprint Backlog}
%TODO: Insert

\subsection{Burndown chart}
% TODO: Insert PIC?

\subsection{Other relevant material}
% TODO: Describe "daily" log files?


\section{Sprint retrospective}

%TODO Describing a sprint retrospective (e.g., as a set of bullets outlining what went well, what went wrong, and how you will improve for the next sprint)

What went well:
\begin{itemize}
	\item We held scrum meetings via Skype, which means we have a chat-log, that everybody can look back at
	\item We setup a shared calender, which updates automaticly, whenever someone puts in a new date 
	\item Using Acunote to track our scrum process
\end{itemize}

\noindent
What went wrong:
\begin{itemize}
	\item We need to be better to commmunicate. Despite our shared calender, we had some hick-ups in meeting times
	\item We did not have a scrum master per se, and as such, nobody took the responibilites of the scrum master 
	\item 
\end{itemize}

\noindent
Steps/measures for improvements in next sprint:
\begin{itemize}
\item We'll chose our scrum master, and this person have got the scrum master responsibilites from day one
\item We should get better at planning, and clearly stating when we are going to meet
\item 
\end{itemize}

\end{document}
